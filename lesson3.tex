

\partabstractfp{\textbf{进程第3天课程摘要}}
\partabstractrp{本单讲述如何根据不同类型的进程分配不同策略的调度算法,调度算法的原理及应用对象。还有如何更改进程的调度策略}
\partabstractlettrine{类}{型,此章讲述不同进程类型消耗的差异以及如何进行进程调度} % the first word of the abstract

\part{进程课第3天}
\chapter{进程种类}
 CPU消耗与IO消耗型
\begin{example*}
  \wdexpbox
  {\caption{ARM的big.LITTLE设计}}
  {采用大核+小核的设计,大核功耗高,运算力强,用于处理CPU消耗性任务,小核功耗低,功耗小,用于处理I/O消耗性任务。实现功耗降低,但处理效果与全是大核处理一致的效果}
\end{example*}

\chapter{进程调度的策略}
\section{RT进程调度}
\subsection{SCHED\_FIFO}
\subsection{SCHED\_RR}


\section{NORMAL进程调度}
\subsection{CFS调度}
\clearpage


\chapter{实践:调整进程优先级}
\section{nice}
%\section(程序运行中调整}
%\section(程序运行中调整}
\clearpage
%%% Local Variables:
%%% TeX-master: "main"
%%% End:
