

\partabstractfp{}
\partabstractrp{}
\partabstractlettrine{F}{AQ,本章记录课间和课后宋老师以及同学们答疑} % the first word of the abstract

\part{进程问题集锦}


\chapter{课后答疑}

以下问题为宋老师在微信答疑群中回答记录,微信群一直存在,问题也不定期提出,此部分内容会随之更新。
\begin{enumerate}
  \item 
\begin{tcolorbox}[colback=green!5,colframe=green!75!black]
\heiti{Q: rps后,非多队列网卡,中断会再给每个核发中断来派发软中断?}
\tcblower
A: 中断只发一个core,这个core自己给别的core发核间中断
\end{tcolorbox}

  \item 
\begin{tcolorbox}[colback=green!5,colframe=green!75!black]
\heiti{Q: rt补丁,是不是只有RT\_FULL支持优先级反转?}
\tcblower
A: 不是,不需要rt补丁就支持优先级继承,早就Merge到了mainline。\\
不叫支持优先级反转,反转是个问题,继承是解决它的方法。反转是个现象,不存在支持不支持。\\
你的问题是错误的。
\end{tcolorbox}


  \item
\begin{tcolorbox}[colback=green!5,colframe=green!75!black]
\heiti{Q: 所以softirq的优先级都是相同的吗?}
\tcblower
A: 不是的,它是一个bit的设置,检查哪个bit被设置,肯定是有先后顺序。挨个检查哪个bit的。不过这个不是关键点。你关心延迟的实时性的时候,你根本就消灭了softirq.
\end{tcolorbox}


\end{enumerate}

%%% Local Variables:
%%% TeX-master: "main"
%%% End:
