

\partabstractfp{}
\partabstractrp{}
\partabstractlettrine{F}{AQ,本章记录课间和课后宋老师以及同学们答疑} % the first word of the abstract

\part{进程问题集锦}

\chapter{课后答疑}


~\\以下问题为宋老师在微信答疑群中回答记录,微信群问题不定期提出,此部分内容会随之更新。
\begin{enumerate}
  \item
\begin{tcolorbox}[colback=green!5,colframe=green!75!black]
\heiti{Q: rps后,非多队列网卡,中断会再给每个核发中断来派发软中断?}
\tcblower
A: 中断只发一个core,这个core自己给别的core发核间中断
\end{tcolorbox}

  \item
\begin{tcolorbox}[colback=green!5,colframe=green!75!black]
\heiti{Q: rt补丁,是不是只有RT\_FULL支持优先级反转?}
\tcblower
A: 不是,不需要rt补丁就支持优先级继承,早就Merge到了mainline。\\
不叫支持优先级反转,反转是个问题,继承是解决它的方法。反转是个现象,不存在支持不支持。\\
你的问题是错误的。
\end{tcolorbox}


  \item
\begin{tcolorbox}[colback=green!5,colframe=green!75!black]
\heiti{Q: 所以softirq的优先级都是相同的吗?}
\tcblower
A: 不是的,它是一个bit的设置,检查哪个bit被设置,肯定是有先后顺序。挨个检查哪个bit的。不过这个不是关键点。你关心延迟的实时性的时候,你根本就消灭了softirq.
\end{tcolorbox}

  \item
\begin{tcolorbox}[colback=green!5,colframe=green!75!black]
\heiti{Q: 内核管理多核好理解,但内核如何管理多CPU?}
\tcblower
A: 异构多os和Linux没关系,那是多os之间的问题,不是Linux管理范畴,几个OS一个通信方法即可。
\end{tcolorbox}

  \item
\begin{tcolorbox}[colback=green!5,colframe=green!75!black]
\heiti{Q: cfs调度单位是task\_struct,那像HMP EAS这些调度器单位是什么?}
\tcblower
A: 调度单元与调度算法无关,你说的调度器和我们说的调度器不一定是一个意思。调度单元就算线程。这个不以任何操作系统,任何算法为转移。
\end{tcolorbox}


  \item
\begin{tcolorbox}[colback=green!5,colframe=green!75!black]
\heiti{Q: 课上说多核可以运行rtos+linux,那是不是单核各自运行一个OS,如何实现,是启动linux后,再启动rtos运行到某个核吗 ?}
\tcblower
A: 多个core单独各玩各的,谁先启动没有讲究,取决于产品,这个和linux没有关系,两个CPU各自运行自己的一套软件。你直接想象成两个电脑就好了。
\end{tcolorbox}



  \item
\begin{tcolorbox}[colback=green!5,colframe=green!75!black]
\heiti{Q: rt的Linux如何对付内存的lazy问题?能否改用内核的api解决用户态内存申请的实时性问题。}
\tcblower
A: 内核和用户态都是用相同的算法来进行调度,内核的调度没有比用户态更优越,用户态的调度策略和优先级比内核高时也可以抢占内核的。内存方面lazy有它的解法,主要是先申请malloc,立即释放,稳住堆;提前调用一个很大的临时变量的函数,稳住栈;提前把线程都创建;mlockall屏蔽交换。这些都有套路,看文档即可。
\end{tcolorbox}



  \item
\begin{tcolorbox}[colback=green!5,colframe=green!75!black]
\hei{Q: 有人说一但调用了\_\_do\_softirq,这个函数不可抢占,这种说法是否正确?}
\tcblower
A: 该怎么抢怎么抢,该抢的你让它抢,你只需保证抢了后的数据不出错即可。别的核还是可以访问你的核正在访问的数据的,tasklet里面该加的锁就得加。保护数据而不是保护过程。
\end{tcolorbox}
\end{enumerate}

%%% Local Variables:
%%% TeX-master: "main"
%%% End:
