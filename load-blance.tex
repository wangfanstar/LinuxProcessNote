
\newcommand{\VerbBar}{|}
\newcommand{\VERB}{\Verb[commandchars=\\\{\}]}
\DefineVerbatimEnvironment{Highlighting}{Verbatim}{commandchars=\\\{\}}
% Add ',fontsize=\small' for more characters per line
\newenvironment{Shaded}{}{}
\newcommand{\AlertTok}[1]{\textcolor[rgb]{1.00,0.00,0.00}{\textbf{#1}}}
\newcommand{\AnnotationTok}[1]{\textcolor[rgb]{0.38,0.63,0.69}{\textbf{\textit{#1}}}}
\newcommand{\AttributeTok}[1]{\textcolor[rgb]{0.49,0.56,0.16}{#1}}
\newcommand{\BaseNTok}[1]{\textcolor[rgb]{0.25,0.63,0.44}{#1}}
\newcommand{\BuiltInTok}[1]{#1}
\newcommand{\CharTok}[1]{\textcolor[rgb]{0.25,0.44,0.63}{#1}}
\newcommand{\CommentTok}[1]{\textcolor[rgb]{0.38,0.63,0.69}{\textit{#1}}}
\newcommand{\CommentVarTok}[1]{\textcolor[rgb]{0.38,0.63,0.69}{\textbf{\textit{#1}}}}
\newcommand{\ConstantTok}[1]{\textcolor[rgb]{0.53,0.00,0.00}{#1}}
\newcommand{\ControlFlowTok}[1]{\textcolor[rgb]{0.00,0.44,0.13}{\textbf{#1}}}
\newcommand{\DataTypeTok}[1]{\textcolor[rgb]{0.56,0.13,0.00}{#1}}
\newcommand{\DecValTok}[1]{\textcolor[rgb]{0.25,0.63,0.44}{#1}}
\newcommand{\DocumentationTok}[1]{\textcolor[rgb]{0.73,0.13,0.13}{\textit{#1}}}
\newcommand{\ErrorTok}[1]{\textcolor[rgb]{1.00,0.00,0.00}{\textbf{#1}}}
\newcommand{\ExtensionTok}[1]{#1}
\newcommand{\FloatTok}[1]{\textcolor[rgb]{0.25,0.63,0.44}{#1}}
\newcommand{\FunctionTok}[1]{\textcolor[rgb]{0.02,0.16,0.49}{#1}}
\newcommand{\ImportTok}[1]{#1}
\newcommand{\InformationTok}[1]{\textcolor[rgb]{0.38,0.63,0.69}{\textbf{\textit{#1}}}}
\newcommand{\KeywordTok}[1]{\textcolor[rgb]{0.00,0.44,0.13}{\textbf{#1}}}
\newcommand{\NormalTok}[1]{#1}
\newcommand{\OperatorTok}[1]{\textcolor[rgb]{0.40,0.40,0.40}{#1}}
\newcommand{\OtherTok}[1]{\textcolor[rgb]{0.00,0.44,0.13}{#1}}
\newcommand{\PreprocessorTok}[1]{\textcolor[rgb]{0.74,0.48,0.00}{#1}}
\newcommand{\RegionMarkerTok}[1]{#1}
\newcommand{\SpecialCharTok}[1]{\textcolor[rgb]{0.25,0.44,0.63}{#1}}
 

% set default fi
\hypertarget{header-n2}{%
\subsection{LINUX下的负载均衡处理对象}\label{header-n2}}

负载均衡是最大化利用CPU资源的方法,要求在有任务(task\_struct,中断,软中断)执行时,所有的CPU都能利用上,不产生有任务处理却有CPU闲置的情况。首先从任务的优先级的角度来看,CPU处理的任务只有下面4种优先级,按高到低依次是:

\begin{quote}
\begin{enumerate}
\def\labelenumi{\arabic{enumi}.}
\item
  中断
\item
  软中断
\item
  处于spinlock等关闭了调度区间的进程
\item
  普通进程
\end{enumerate}
\end{quote}

以上优先级的任务在linux处理规则如下:

\begin{quote}
1、中断不可以嵌套中断,在2.6版本后,处于中断区间再次发生中断时,会等到前一个中断执行结束后再进行处理下一个中断。

2、中断可以唤起软中断

3、软中断可以唤起中断

4、CFS等调度算法只处理普通进程和普通进程之间的调度,不涉及中断,软中断,及关闭了调度的进程。具体表现如下:

\begin{itemize}
\item
  如果CPU在处理1,2,3优先级的任务时,不受调度算法的调度,只有等处理完1,2,3优先级的任务后才会再由调度算法调度。
\item
  如果CPU在处理4普通进程的任务时,高优先级的中断和软中断可以直接抢占普通进程,不用等调度算法调度。
\end{itemize}
\end{quote}

\hypertarget{header-n39}{%
\subsection{中断负载均衡}\label{header-n39}}

在TOP命令中,cpu时间占用中有一列是\textbf{hi}和\textbf{si},分别对应中断和软中断。说明cpu时间除了在task\_struct上,还有可能花在中断和软中断,当网络流量比较大进,cpu花在中断和软中断的时间比较大,可以考虑中断负载均衡。

分配IRQ到某个CPU,掩码01代表CPU0,02代表CPU1,04代表CPU2,08代表CPU3

\begin{Shaded}
\begin{Highlighting}[]
\CommentTok{#此命令将中断145分配到CPU0上处理}
\NormalTok{[}\ExtensionTok{root@boss}\NormalTok{ ~] }\CommentTok{# echo 01 > /proc/irq/145/smp_affinity}
\NormalTok{[}\ExtensionTok{root@boss}\NormalTok{ ~] }\CommentTok{# cat /proc/irq/145/smp_affinity}
			\ExtensionTok{00000001}
\end{Highlighting}
\end{Shaded}

\hypertarget{header-n47}{%
\subsection{软中断负载均衡-\/-rps}\label{header-n47}}

有时候有的网卡只有一个队列,一个队列的中断只能分配到一个核,Linux设计是一个核上抛出的软中断只能在同一个核上执行,cpu
0上的中断抛出一个软中断,tcp/ip协议栈也只能在cpu
0的软中断上处理。google在linux内核里面加入了rps补丁,其作用是尽管中断是在一个cpu核上,但tcp/ip协议处理的软中断可以在多个核上进行处理。rps的原理是收到软中断后通过核间中断的方法给另外的核发中断,让其他核处理软中断,从而支持单队列情况下的网络负载均衡。

\begin{Shaded}
\begin{Highlighting}[]
\CommentTok{#rps 使能方法,除了CPU 0 外都参与TCP/IP协议栈}
\BuiltInTok{echo}\NormalTok{ fffe }\OperatorTok{>}\NormalTok{ /sys/class/net/eth1/queues/rx-0/rps_cpus}

\CommentTok{#查看softirqs}
\ExtensionTok{wangfan@wangfan-VirtualBox}\NormalTok{:~$ cat /proc/softirqs }
                    \ExtensionTok{CPU0}\NormalTok{       CPU1       }
          \ExtensionTok{HI}\NormalTok{:          0          2}
       \ExtensionTok{TIMER}\NormalTok{:    6841572    6725135}
      \ExtensionTok{NET_TX}\NormalTok{:          1      17644}
      \ExtensionTok{NET_RX}\NormalTok{:        679     224896}
       \ExtensionTok{BLOCK}\NormalTok{:      61380     180153}
    \ExtensionTok{IRQ_POLL}\NormalTok{:          0          0}
     \ExtensionTok{TASKLET}\NormalTok{:         15       7834}
       \ExtensionTok{SCHED}\NormalTok{:    3148547    3016778}
     \ExtensionTok{HRTIMER}\NormalTok{:          0          0}
         \ExtensionTok{RCU}\NormalTok{:     747890     885505}
\end{Highlighting}
\end{Shaded}

\hypertarget{header-n51}{%
\subsubsection{利用rps解决cpu占用率高的问题}\label{header-n51}}

宋老师关于爱立信工程师的问题处理,爱立信的工程师在服务器上写了个软件发现16核有2个核占用率很高,但其他核都很闲,top命令查看发现\textbf{hi}和\textbf{si}很高,说明cpu大部分时间在处理中断和软中断,而不是处理task\_struct。解决方法是登录机器后敲命令echo
ffff 到rps,cpu占用率降了下来,效果很明显。

\hypertarget{header-n56}{%
\subsection{进程间(task\_struct)负载均衡}\label{header-n56}}

\hypertarget{header-n57}{%
\subsubsection{linux负载均衡算法原则}\label{header-n57}}

linux下所有CPU核会进行分布式的PUSH和PULL操作,当CPU核空闲时会向周围的核PULL任务来执行,CPU核本身在执行任务时也会PUSH任务到其他核。

每个核执行同样的负载均衡算法,负载均衡包括RT任务的负载均衡和普通任务的负载均衡

\begin{itemize}
\item
  RT任务的负载均衡算法是将N个优先级最高的RT分布到N个核

  pull\_rt\_task(); push\_rt\_task()
\item
  普通任务负载均衡有三种:IDLE式负载均衡,周期性负载均衡,FORK和EXEC式负载均衡

  周期性负载均衡:时钟tick的时间点上CPU核查询自己是否很闲,周围核是否很忙,是则用PULL将周围核的任务拉过来处理。

  IDLE式负载均衡:当CPU核在IDLE时会查询周围核是否在忙,如果旁边核比较忙时,自动PULL旁边核的task\_struct任务来执行。

  FORK和EXEC式负载均衡:FORK和EXEC创建一个新的进程时,Linux会自动找一个最闲的核将FORK和EXEC新创建出的进程放在上面处理。

  以上处理由核与核之间分布式负载均衡处理是自动进行的。
\end{itemize}

\hypertarget{header-n79}{%
\subsubsection{由于负载均衡导致cpu占用率200\%的示例}\label{header-n79}}

在一个程序中起2个进程,每个进程都在做CPU消耗型操作(代码中调用while(1)死循环),在进程执行的过程中会自动将进程分配到2个CPU核上进行处理。可以通过查看CPU占用率和时间占用情况来验证。分配到两个CPU核后,CPU占用率会上升到200\%,用time计算程序的占用时间,真实时间是系统时间的一半,因为系统时间是单独统计每个CPU核上占用的时间,2个CPU核上会统计2次,显示的结果就是系统时间是真实时间的2倍。

\hypertarget{header-n82}{%
\subsubsection{设置进程在指定CPU上运行}\label{header-n82}}

要设置进程在指定CPU上运行,在代码里可以通过调用相关API实现,也可以直接在BASH中通过taskset命令实现

\begin{Shaded}
\begin{Highlighting}[]
\CommentTok{//设置CPU task affinity api}
\PreprocessorTok{#include}\ImportTok{<pthread.h>}\PreprocessorTok{ }\CommentTok{//注意<pthread.h>包含<sched.h>}
\DataTypeTok{int}\NormalTok{ pthread_setaffinity_np(pthread_t thread,}\DataTypeTok{size_t}\NormalTok{ cpusetsize,}\DataTypeTok{const}\NormalTok{ cpu_set_t *cpuset);}
\DataTypeTok{int}\NormalTok{ pthread_getaffinity_np(pthread_t thread,}\DataTypeTok{size_t}\NormalTok{ cpusetsize, cpu_set_t *cpuset);}
\DataTypeTok{int}\NormalTok{ sched_setaffinity(pid_t pid, }\DataTypeTok{size_t}\NormalTok{ cpusetsize, cpu_set_t *mask);}
\DataTypeTok{int}\NormalTok{ sched_getaffinity(pid_t pid, }\DataTypeTok{size_t}\NormalTok{ cpusetsize, cpu_set_t *mask);}
\end{Highlighting}
\end{Shaded}

\begin{Shaded}
\begin{Highlighting}[]
\CommentTok{# taskset在bash下的使用方法}
\CommentTok{# 命令行形式设置CPU亲核性}
\ExtensionTok{taskset}\NormalTok{ [options] mask command [arg]...}
\ExtensionTok{taskset}\NormalTok{ [options] -p [mask] pid}

\ExtensionTok{PARAMETER}
    \ExtensionTok{mask}\NormalTok{ : cpu亲和性,当没有-c选项时, 其值前无论有没有0x标记都是16进制的,}
\NormalTok{        当有}\ExtensionTok{-c}\NormalTok{选项时,其值是十进制的.}
    \ExtensionTok{command}\NormalTok{ : 命令或者可执行程序}
    \ExtensionTok{arg}\NormalTok{ : command的参数}
    \ExtensionTok{pid}\NormalTok{ : 进程ID,可以通过ps/top/pidof等命令获取}


\ExtensionTok{OPTIONS}
    \ExtensionTok{-a}\NormalTok{, --all-tasks (旧版本中没有这个选项)}
\NormalTok{        这个选项涉及到了}\ExtensionTok{linux}\NormalTok{中TID的概念,他会将一个进程中所有的TID都执行一次CPU亲和性设置.}
        \ExtensionTok{TID}\NormalTok{就是Thread ID,他和POSIX中pthread_t表示的线程ID完全不是同一个东西.}
        \ExtensionTok{Linux}\NormalTok{中的POSIX线程库实现的线程其实也是一个进程(LWP),这个}\ExtensionTok{TID}\NormalTok{就是这个线程的真实PID.}
       \ExtensionTok{-p}\NormalTok{, --pid}
\NormalTok{              操作已存在的}\ExtensionTok{PID}\NormalTok{,而不是加载一个新的程序}
       \ExtensionTok{-c}\NormalTok{, --cpu-list}
\NormalTok{              声明}\ExtensionTok{CPU}\NormalTok{的亲和力使用数字表示而不是用位掩码表示. 例如 0,5,7,9-11.}
       \ExtensionTok{-h}\NormalTok{, --help}
              \ExtensionTok{display}\NormalTok{ usage information and exit}
       \ExtensionTok{-V}\NormalTok{, --version}
              \ExtensionTok{output}\NormalTok{ version information and exit}
\ExtensionTok{USAGE}

    \ExtensionTok{1}\NormalTok{) 使用指定的}\ExtensionTok{CPU}\NormalTok{亲和性运行一个新程序}
      \ExtensionTok{taskset}\NormalTok{ [-c] mask command [arg]...}
\NormalTok{        举例:使用}\ExtensionTok{CPU0}\NormalTok{运行ls命令显示/etc/init.d下的所有内容 }
          \ExtensionTok{taskset}\NormalTok{ -c 0 ls -al /etc/init.d/}

    \ExtensionTok{2}\NormalTok{) 显示已经运行的进程的}\ExtensionTok{CPU}\NormalTok{亲和性}
      \ExtensionTok{taskset}\NormalTok{ -p pid}
\NormalTok{        举例:查看}\ExtensionTok{init}\NormalTok{进程(PID=1)的}\ExtensionTok{CPU}\NormalTok{亲和性}
          \ExtensionTok{taskset}\NormalTok{ -p 1}

    \ExtensionTok{3}\NormalTok{) 改变已经运行进程的}\ExtensionTok{CPU}\NormalTok{亲和力}
        \ExtensionTok{taskset}\NormalTok{ -p[c] mask pid}
\NormalTok{        举例:打开}\ExtensionTok{2}\NormalTok{个终端,在第一个终端运行top命令,第二个终端中}
\NormalTok{          首先运行:[}\ExtensionTok{~}\NormalTok{]# }\FunctionTok{ps}\NormalTok{ -eo pid,args,psr }\KeywordTok{|} \FunctionTok{grep}\NormalTok{ top }\CommentTok{#获取top命令的pid和其所运行的CPU号}
\NormalTok{          其次运行:[}\ExtensionTok{~}\NormalTok{]# }\ExtensionTok{taskset}\NormalTok{ -cp 新的CPU号 pid       #更改top命令运行的CPU号}
\NormalTok{          最后运行:[}\ExtensionTok{~}\NormalTok{]# }\FunctionTok{ps}\NormalTok{ -eo pid,args,psr }\KeywordTok{|} \FunctionTok{grep}\NormalTok{ top }\CommentTok{#查看是否更改成功}

\ExtensionTok{PERMISSIONS}
\NormalTok{        一个用户要设定一个进程的}\ExtensionTok{CPU}\NormalTok{亲和性,如果目标进程是该用户的,则可以设置,如果是其他用户的,则会设置失败,提示 Operation not permitted.当然root用户没有任何限制.}
\NormalTok{        任何用户都可以获取任意一个进程的}\ExtensionTok{CPU}\NormalTok{亲和性.}
\end{Highlighting}
\end{Shaded}

\hypertarget{header-n91}{%
\subsection{给进程指定比例的CPU负载-\/-cgroup}\label{header-n91}}

当前的程序是按程序的需要来占用cpu的,这样可能会出现一些问题,比如用户A和B在同一个服务器上,如果A开的线程比B多,可能导致A一直占用cpu,B因为线程少,占用的权重比例少而得不到cpu。于是我们想要一个分群的概率,让A和B各占有50\%的CPU,不管A线程有多少,最多只能占50\%的CPU,这样保证B即使线程数量少,也可以得到足够的CPU来运行。同样的道理类似于计费的网络带宽,可以根据用户缴费的情况分配CPU,如果未交费,就算CPU空闲也不分配CPU给用户。

\hypertarget{header-n94}{%
\subsubsection{cgroup使用方法}\label{header-n94}}

cgroup主要是设置以下3个属性,在\texttt{/sys/fs/cgroup/cpu}目录mkdir一个group后,会出现很多属性文件,我们主要通过以下属性来查询和设置。

\begin{quote}
cgroup.procs :将进程号echo进去

cpu.cfs\_period\_us : 默认是100000 基准时间100ms

cpu.cfs\_quota\_us :
配额默认是-1最大值,设置可以比100000大,它与period的比例表示gruop内线程最高可占cpu的比例

cpu.shares :
权重,默认是1024,调节cpu.shares可以调节不同group的cpu占用率。
\end{quote}

\begin{Shaded}
\begin{Highlighting}[]
\CommentTok{# 1 cd /sys/fs/cgroup/cpu 目录创建不同的group}
\CommentTok{# 2 mkdir A  创建 group A}
\CommentTok{# 3 mkdir B  创建 group B}
\CommentTok{# 4 /sys/fs/cgroup/cpu/A  echo 3582 > cgroup.procs 将进程3582加入A group}
\CommentTok{# 5 /sys/fs/cgroup/cpu/B  echo 3581 > cgroup.procs 将进程3581加入B group}
\CommentTok{# 6 /sys/fs/cgroup/cpu/A  echo 50000 > cpu.cfs_quota_us  设置A group 权重为 50% cpu, A内线程的cpu占用率最高不超过50%}
\end{Highlighting}
\end{Shaded}

\hypertarget{header-n111}{%
\subsubsection{安卓的cgroup设计}\label{header-n111}}

安卓5.0之后的版本用到了cgroup,安卓早期版本所有进程都采用调度算法公平调度,最新版本把进程分为前台交互进程和后台非交互进程,前台的权重是1024,后台的权重是52,这样前台可以得到更多的CPU,用于提高前台程序的响应。

